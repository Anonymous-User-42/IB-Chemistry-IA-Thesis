            
%\subimport{./}{FDGraph}

%	\subimport{./}{FDGraph Mod 1}

%\textit{The above graph is the \textbf{scatter plot} of all \textbf{data points} within the \textbf{data set} of \textbf{experimentally collected values} with the \textbf{cubic regression} of the same. A \textbf{cubic regression} was utilized to plot the \textbf{line of best fit}, because a regression creates an \textbf{accurate model} of the system according to the data points and is a \textbf{mathematical function of each and every data point}. This was used rather than plotting an simple line between the range of the \textbf{maximum and minimum} values of the data points, as it would be similar to guesswork.}

%	{The above graph is the \textbf{line graph} based on the averages of the \textbf{data points} of the \textbf{experimentally collected values} The error bars represent the error in the measurement of both \textbf{temperature} and the \textbf{TDS} readings.}

%	{Using the data points from this data set plotted on the above graph, a cubic regression model was created an plotted to \textbf{effectively study the trends} of the behavior/phenomenon of the solubility of Oxalic acid in various alcohols with change in temperature.}

\subimport{./}{RLGraph}

{The \textbf{cubic regression} of all \textbf{data points} within the given \textbf{data set} of \textbf{experimentally collected values} is depicted in the graph above. Because a regression generates an appropriate model of the system based on the data points and is a \textbf{mathematical function} of each and every data point, a cubic regression was used to depict the \textbf{line of best fit}. This was employed instead of showing a simple line between the range of the data point's \textbf{maximum and minimum} values because it would have been similar to guesswork.}

{\textbf{Note}: The \textbf{cubic regression} was plotted to only \textbf{study the trends} (gradation) of the \textbf{changes in the solubility of Oxalic acid in various alcohols with changes in temperature}. The cubic regression model does \textbf{not} accurately model the \textbf{behavior of the phenomenon} of the solubility in alcohol's, though it can \textbf{predict the trends} at a given \textbf{temperature range} (domain of temperature in the study).}

{\textbf{Note}: Because of resource constraints (limited data set), the \textbf{cubic regression model} was used rather than a \textbf{quintic} or \textbf{higher order regression model}, although the portion of the model we're investigating \textbf{behaves exactly} like a higher order regression model over temperature when constrained to the range of 20 $^\circ$ C to 60 $^\circ$ C.}

{On effectively studying and analyzing the above graph (regression graph) with scrutiny, we observe that,}

	\begin{itemize}
		\item {The solubility of all alcohols generally \textbf{increases} with \textbf{increase in temperature}, till a cutoff point which is the point at which the \textbf{trend of the model reverses}}
		\item {The cutoff point generally increases with increase in temperature over the hierarchy of the homologous series of alcohols, ie. \textbf{The cutoff point in alcohols generally increases with respect to temperature as the length of the hydrocarbon chain increases}}
		\item {Alcohols that are \textbf{odd} in the hierarchy of the homologous series of alcohols, first tend to \textbf{increase} and then \textbf{decrease} (this pattern may sequentially continue), this pattern can be evident by studying the trends of \textbf{Methanol}, \textbf{Propanol} and \textbf{Pentanol}}
		\item {Alcohols that are \textbf{even} in the hierarchy of the homologous series of alcohols, first tend to \textbf{decrease} and then \textbf{increase} (this pattern may sequentially continue), this pattern can be evident by studying the trends of \textbf{Ethanol} and \textbf{Butanol}}
		\item {The solubility of Oxalic acid in various alcohols (in the order of the hierarchy of the homologous series of alcohols), generally \textbf{decreases}, ie. \textbf{The solubility in alcohols generally decreases as the length of the hydrocarbon chain increases}}
		\item {The solubility of Oxalic acid in a particular alcohol, generally \textbf{increases} with increase in temperature}
		\item {The solubility of Oxalic acid in various alcohols (in the order of the hierarchy of the homologous series of alcohols), generally \textbf{decreases} with increase in temperature, ie. \textbf{The solubility in alcohols generally decreases with respect to temperature as the length of the hydrocarbon chain increases}}
	\end{itemize}

%	%	%	%	%
            
%\textit{Upon close visual observation, we see that the difference in the plotted values of that of the \textbf{simulation} and \textbf{experimental values} of the \textbf{drag force} versus \textbf{temperature} from the \textbf{F-T} graph are very \textbf{minute} to the extent that it would be right to say and consider that the experimental values are both \textbf{accurate} and \textbf{precise} in relation to that of the \textbf{literature/theoretical/simulation values}.}
        
%\textit{It is evident from studying the system \textbf{numerically} and \textbf{graphically} that the system exhibits an \textbf{linear decay} with time for the parameter that is being investigated. This points out to \textbf{one conclusion}, that the \textbf{parameter of the system undergoes linear decay not exponential decay}, according to their various \textbf{inert energies}.}
	 
%\textit{This \textbf{partially contradicts} the \textbf{initial hypothesis} laid out prior to beginning the investigation as if, the system had to model \textbf{similarly} to the inverse exponential, then the linear behavior that we observe would have not existed, but we see that this is not the case.}        
        

%\textit{It would be right to say that the \textbf{initial hypothesis} that was laid out prior, beginning the experimentation was \textbf{partially correct} and is a \textbf{valid statement}.}        
    
    

